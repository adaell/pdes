\documentclass{article}
\usepackage{amsmath}
\usepackage{amssymb}
\usepackage{multicol}
\usepackage[backend=biber]{biblatex}
\addbibresource{readme.bib}

\begin{document}
\title{Notes}

\section{Parabolic PDEs}

\subsection{\texttt{diff\_adv\_2D.py}} 

Calculates the solution to the two-dimensional diffusion-advection/heat-convection equation
\begin{equation*}
\frac{\partial u}{\partial t} = D(x,y) \hspace{0.1cm} \nabla^{2} u  - \mathbf{\hat{v}} \cdot \nabla u + q(x,y,t),
\end{equation*}
using cartesian coordinates on a uniform mesh, on the domain $\Omega$ bound by $x \in [0,L_x]$ and $y \in [0,L_y]$, where
\begin{equation*}
\mathbf{\hat{v}} = v_x \mathbf{\hat{\imath}} + v_y \mathbf{\hat{\jmath}}
\end{equation*}
describes the direction of advection/convection, $D(x,y)$ is the diffusivity/conductivity and $q(x,y,t)$ is a source/sink term. The program assumes Robin boundary conditions
\begin{equation*}
a \hspace{0.1cm} u(x,y,t) + b \hspace{0.1cm} \frac{\partial u}{\partial n} = g(x,y)
\end{equation*}
where $a$ and $b$ are real scalars, $g(x,y)$ is an arbitrary function on the boundary and $\partial n$ denotes differentiation in the direction of a normal to the boundary. The initial condition $u_0(x,y)=u(x,y,0)$ can be any real valued function. The solution is calculated using the finite difference method.

\subsection{\texttt{mc\_stefan.py}}
Calculates the solution to the classical Stefan problem
\begin{align*}
\frac{\partial u}{\partial t} &= \alpha \hspace{0.1cm} \frac{\partial^2 u}{\partial x^2}, \hspace{0.5cm} 0 < x < s(t), \hspace{0.2cm} t > 0 \\
u(0,t)&=1,\\
u(x,0)&=0,\\
u(s(t),t)&=T_m\\
s(0)&=0\\
L \hspace{0.1cm} \rho \hspace{0.1cm} \frac{ds}{dt} &= k \hspace{0.1cm} \frac{\partial u}{\partial x} %\|_{x=s(t)
\end{align*}
using a monte carlo, where $u(x,t)$ is the temperature, $s(t)$ is the position of the moving boundary, $\alpha$ is the thermal diffusivity, $T_m$ is the melting temperature, $L$ is the latent heat, $\rho$ is the density and $k$ is the thermal conductivity. The solution is calculated on the domain $\Omega$ bound by $x \in [0,L_x]$ and $t > 0$. The code models the free boundary using the method proposed by Stoor \cite{stoor}. 
\newline 

The answer is compared to the analytical solution derived by Neumann\footnote{A similarity solution derived using the variable $\xi=\frac{x}{\sqrt{t}}$.}
\begin{align*}
u(x,t)&=1-\frac{\mathrm{erf} \left( \frac{x}{2\sqrt{t}} \right) }{ \mathrm{erf} \left( \lambda \right)}, \\
s(t)&=2 \lambda \sqrt{t},
\end{align*}
where $\lambda$ satisfies
\begin{equation}
\beta \sqrt{\pi}\lambda \mathrm{e}^{\lambda^2} \mathrm{erf} \left( \lambda \right) = 1,
\end{equation}
where $\beta$ is the Stefan number.

\section{Hyperbolic PDEs}

\subsection{\texttt{wave\_2D.py}} 
Calculates the solution to the two-dimensional acoustic wave equation
\begin{equation*}
\frac{\partial^2 p}{\partial t^2} + \nu(x,y) \hspace{0.1cm} \frac{\partial p}{\partial t} = c^2 \hspace{0.1cm} \nabla^{2} u + q(x,y,t),
\end{equation*}
using cartesian coordinates on a uniform mesh, on the domain $\Omega$ bound by $x \in [0,L_x]$ and $y \in [0,L_y]$, where $\nu(x,y)$ is a damping term, $c$ is the speed of sound and $q(x,y,t)$ is a source/sink term. The program assumes Robin boundary conditions
\begin{equation*}
a \hspace{0.1cm} u(x,y,t) + b \hspace{0.1cm} \frac{\partial u}{\partial n} = g(x,y),
\end{equation*}
where $a$ and $b$ are real scalars, $g(x,y)$ is an arbitrary function on the boundary and $\partial n$ denotes differentiation in the direction of a normal to the boundary. The initial pressure $p_0(x,y)=p(x,y,0)$ can be any real valued function. The solution is calculated using the finite difference method.


\subsection{\texttt{wave\_2D\_PML.py}}
Calculates the solution to the two-dimensional acoustic wave equation
\begin{equation*}
\begin{aligned}
\frac{\partial \mathbf{\hat{v}}}{\partial t} &= - \frac{1}{\rho} \nabla p,\\
\frac{\partial p}{\partial t} + \nu(x,y) \hspace{0.1cm} p &= -c^2 \rho \nabla \cdot \mathbf{\hat{v}} + q(x,y,t),
\end{aligned}
\end{equation*}
where $\mathbf{\hat{v}}(x,y,t)$ describes the velocity at each point in the mesh and $p(x,y,t)$ describes the pressure at each point in the mesh. Solution is calcualted using cartesian coordinates and a uniform mesh, on the domain $\Omega$ bound by $x \in [0,L_x]$ and $y \in [0,L_y]$, where $\nu(x,y)$ is a damping term, $c$ is the speed of sound and $q(x,y,t)$ is a source/sink term. The program assumes Robin boundary conditions
\begin{equation*}
a \hspace{0.1cm} u(x,y,t) + b \hspace{0.1cm} \frac{\partial u}{\partial n} = g(x,y),
\end{equation*}
where $a$ and $b$ are real scalars, $g(x,y)$ is an arbitrary function on the boundary and $\partial n$ denotes differentiation in the direction of a normal to the boundary. The initial pressure $p_0(x,y)=p(x,y,0)$ can be any real valued function. The solution is calculated using the finite difference method. Absorbing boundary conditions can optionally be turned on or off to simulate far-field conditions. Absorbing boundary conditions are implemented by stretching the coordinates of the governing equations into the complex domain in Fourier-transform-space using the method proposed by Berenger \cite{Berenger1994}.


\section{Elliptical PDEs}
\subsection{\texttt{helmholtz.py}} 
Calculates the solution to the nonhomogenous Helmholtz equation
\begin{equation*}
(\nabla^2 + k(x,y)^2) \hspace{0.1cm} f(x,y,t) = \psi(x,y),
\end{equation*}
using cartesian coordinates on a uniform mesh, on the domain $\Omega$ bound by \newline $x \in [0,L_x]$ and $y \in [0,L_y]$, where $k(x,y)$ and $\psi(x,y)$ are real valued functions. The program assumes Robin boundary conditions
\begin{equation*}
a \hspace{0.1cm} u(x,y,t) + b \hspace{0.1cm} \frac{\partial u}{\partial n} = g(x,y),
\end{equation*}
where $a$ and $b$ are real scalars, $g(x,y)$ is an arbitrary function on the boundary and $\partial n$ denotes differentiation in the direction of a normal to the boundary. The solution is calculated using the finite difference method. \newline
\newline
\texttt{keywords: Laplace's equation, Poisson's equation.}

\section{References}
\printbibliography[title={Articles},type=article,sorting=nyt,heading=subbibliography]

\end{document}
